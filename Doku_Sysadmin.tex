\documentclass[12pt]{article}
 \usepackage[german]{babel}
 \usepackage[utf8]{inputenc}
 \usepackage{graphicx}
 \author{Denis Herdt, Almin Causevic}
 \title{ROS auf dem Rapberry Pi}
 \setlength{\parindent}{0pt}                   % Einrueckung 1. Zeile eines Absatzes
 \setlength{\parskip}{5pt plus 2pt minus 1pt}  % Abstand zwischen Absaetzen
 \frenchspacing
 \sloppy
 
 
\begin{document}

\begin{figure}[h]

\includegraphics[width=4cm]{hs-logo.jpg}
\end{figure}
Fakultät für Elektrotechnik und Informatik

\vspace{3cm}

\begin{center}

{\bf \huge ROS auf dem Raspberry Pi}
\vspace{4cm}

14 November, 2013
\vspace{1cm}

Systemadministration Projekt in Angewandter Informatik \\
von Denis Herdt und Almin Causevic

\end{center}

\pagebreak

\tableofcontents

\pagebreak

\section{Einleitung}
1-2 Seiten
\subsection{Motivation}

Roboter Lab Anwendung von ROS und Möglichkeit, Roboter zu nutzen

Interesse der Lab-Crew an Umsetzung ROS mit Raspberry Pi

Kombination Hard-/Software

Einarbeitung in Kommunikations Framework ROS

Mit Hardware in Berührung zu kommen

Konfiguration mit Linux

Nodes nun mit Hardware (Pi) realisierbar

\subsection{Zielstzung}

Recherche über Linux Systeme auf Raspberry Pi
Passendes Linux System auf Raspberry Pi aufsetzen					
Recherche über ROS									
geeigneten Roboter wählen (Recherche Verfügbarkeit, Komponenten etc.)	
eventuell Komponenten beschaffen (Wlan Stick, etc.)				
ROS auf Raspberry Pi aufsetzen		
Wlan-Netzwerk zwischen 2 ROS-kompatiblen Systemen herstellen
Netzwerk mithilfe von Testausgaben überprüfen \\

{\bf optional}
Roboter durch einen Raspberry Pi mithilfe von ROS steuern
Bild einer Webcam durch ROS auf einen Bildschirm streamen

\subsection{Eigene Leistung}

anstatt Linux PC wird Pi benutzt

\subsection{Aufbau der Arbeit}

Recherche über Linux Systeme auf Raspberry Pi
Passendes Linux System auf Raspberry Pi aufsetzen					
Recherche über ROS									
geeigneten Roboter wählen (Recherche Verfügbarkeit, Komponenten etc.)	
eventuell Komponenten beschaffen (Wlan Stick, etc.)				
ROS auf Raspberry Pi aufsetzen		
Wlan-Netzwerk zwischen 2 ROS-kompatiblen Systemen herstellen
Netzwerk mithilfe von Testausgaben überprüfen \\

{\bf optional}
Roboter durch einen Raspberry Pi mithilfe von ROS steuern
Bild einer Webcam durch ROS auf einen Bildschirm streamen

\section{Grundlagen}
6-8 Seiten
Hardware, die benutzt wird:

Raspberry Pi (Akku, Wlan)
Volksbot
Smartphone
Kabelz

Software, die benützt wird:

ROS (catkin)
Linux-System
Software Smartphone (App Steffen)

\section{Problem}

Hauptproblem??

\subsection{Verwandte Arbeiten}

Steffen, Marc
ROS-Tut

\section{Anforderungen}

Datenpakete übers ROS Netzwerk verschicken kann
Beliebige Datentypen
Beliebige Linux-Systeme
Überprüfung davon Log-Dateien, Kontrollausgaben etc..
robuste und zuverlässige Kommunikation möglich
modulares und flexibles Netzwerk
Pakete (repos) sollen leicht in Projekte eingebunden werden

optional:
Roboter soll bei Steuerung in richtige Richtungen fahren
Roboter soll sofort auf Richtungsanweisungen reagieren
Robotersteuerung Sensibilität sollte einstellbar sein
Steuerung über Wlan
Raspberry Pi auf Roboter ohne Kabelzgewirr
Node des Raspberry starte automatisch mit Raspbian
Leichte Bedienbarkeit des Roboters

optional:
Streaming-Bild ruckelfrei bearbeitet werden
soll unkonvertiert in Echtzeit übertragen werden
soll an externem Bildschirm ausgegeben werden

\section{Lösungsvorschläge}

Netzwerk wird mit ROS-Master aufgesetzt, über topics und nodes
verschicken und empfangen von Paketen 

beliebige Datentypen durch msg und srv Funktion realisierbar

ROS unterstüzt beliebige Linux Systeme..

Horchen an Topics mit echo + Ausgabe im Code mithilfe ROS-Stream Fkt.

Begrenzung der Anzahl der Datenübertragung
Verschickte und erwartete Datentypen müssen stimmen
Richtige Konfiguration des Masters und Clients

durch ROS und catkin gegeben 

durch catkin gegeben

opt:
abhängig vom richtigen Code der Motorsteuerung

abhängig von Raspberry Pi Hardware und effizientem Code und
guter Hardware

abhängig vom Code

Hardware nötig Wlan-Stick, Akku-Pack 

sollte in Linux beim booten konfiguriert werden

gegeben durch gute Smartphone App

opt:
abhängig von Raspberry Pi Hardware und Übertragungs-Codierung

abhängig vom Codex Hardware Komponenten

nötige Hardware und Netzwerk legen

\section{Bewertung der Lösungen}

alles gut

\section{Implementation}

ROS Netzwerk (topics, nodes, catkin...)
Linux SD Karte Pi
Datenübertragung an Volksbot
Konfiguration Motorsteuerung-Paket und Compilieren
Kontrollausgaben 
Netzwerkkomponenten zusammen arbeiten lassen (Hard/-Software)
Wlan -> Teilproblem
Kamera -> Teilproblem Zeit


\section{Fazit}
1 Seite
\subsection{Zusammenfassung}

sehr knapper Zeitrahmen
hohe Anforderungen für diesen Zeitraum
interessantes Projekt
anspruchsvolles Projekt
viel über Linux und Ros und Hardware und Netzwerken gelernt
Spaß

\section{Ausblick}

Benutzung des Pi's als Hardware Nodes Lab
Benutzung des Pi's für kleine Roboterprojekte Lab
Immer mehr binary packages für ROS Pi verfügbar -> mehr Möglichkeiten
z.B.: GPIO Interface -> Knöpfe Kaffemaschine, Relays, alle möglichen Signale
auf Hardwareebene

\section{Eigene Leistung}

anstatt Linux PC wird Pi benutzt

\end{document}