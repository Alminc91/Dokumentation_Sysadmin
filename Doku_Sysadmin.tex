\documentclass[12pt]{article}
 \usepackage[german]{babel}
 \usepackage[utf8]{inputenc}
 \usepackage{graphicx}
 \author{Denis Herdt, Almin Causevic}
 \title{ROS auf dem Rapberry Pi}
 \setlength{\parindent}{0pt}                   % Einrueckung 1. Zeile eines Absatzes
 \setlength{\parskip}{5pt plus 2pt minus 1pt}  % Abstand zwischen Absaetzen
 \frenchspacing
 \sloppy
 
 
\begin{document}
%4-20 Seiten
\begin{figure}[h]

\includegraphics[width=4cm]{hs-logo.jpg}
\end{figure}
Fakultät für Elektrotechnik und Informatik

\vspace{3cm}

\begin{center}

{\bf \huge ROS auf dem Raspberry Pi}
\vspace{4cm}

14 November, 2013
\vspace{1cm}

Systemadministration Projekt in Angewandter Informatik \\
von Denis Herdt und Almin Causevic

\end{center}

\pagebreak

\tableofcontents

\pagebreak

\section{Einleitung}
1-2 Seiten
\subsection{Motivation}

Wir haben uns für dieses Thema entschieden, weil wir mit Hard- und Software arbeiten wollten. 

Wir haben mitbekommen, wie umständlich z.B. der Roboter ''Volksbot'' im Robotiklabor der Hochschule mit einem zusätzlichen, auf dem Roboter platzierten Laptop gesteuert werden muss, der unnötig Platz und Gewicht beansprucht. Deshalb haben wir uns entschlossen, dieses Problem mithilfe des kleinen und leichten Raspberry Pi zu lösen.
Die Idee stieß auch auf großes Interesse bei den Labormitarbeitern und bringt viel praktischen Nutzen.

Unser Ziel möchten wir mithilfe des Software-Frameworks ROS (Robot Operating System) realisieren.
Für ROS haben wir uns entschieden, weil es zur Zeit die aktuellste und inovativste Methode ist, ein flexibles Netzwerk von Komponenten für z.B. die Robotersteuerung zu realisieren. Es wird weltweit eingesetzt und könnte auch in Zukunft sehr nützlich werden.

Durch die Kombination mit der Mächtigkeit und Flexibilität von ROS und der geringen Größe und dem niedrigen Energieverbrauch des Pi eröffnen sich außerdem neue und effizientere Einsatzmöglichkeiten.

Sehr gut finden wir auch die Tatsache, dass wir mit Linux arbeiten können und dieses Projekt viel Konfiguration des Betriebssystems voraussetzt, da wir auch gerne mehr Praxiserfahrung mit Linux haben möchten.

%Roboter Lab Anwendung von ROS und Möglichkeit, Roboter zu nutzen

%Interesse der Lab-Crew an Umsetzung ROS mit Raspberry Pi

%Kombination Hard-/Software

%Einarbeitung in Kommunikations Framework ROS

%Mit Hardware in Berührung zu kommen

%Konfiguration mit Linux

%Nodes nun mit Hardware (Pi) realisierbar

\subsection{Zielsetzung}


Unser Hauptziel besteht darin, ROS auf dem Raspberry Pi aufzusetzen und zu nutzen. Dieses Ziel wird wie folgt unterteilt:\\

Zunächst möchten wir ein passendes, auf Linux basierendes Betriebssystem auswählen und auf dem Raspberry Pi aufsetzen. In diesem Fall haben wir uns für Raspbian entschieden, da es zusätzlich am kopatibelsten zu ROS ist.

Als nächstes wählen wir einen geeigneten Roboter aus. Die Kriterien dafür erarbeiten wir uns aus einer vorangegangenen ROS-Recherche.

Im Anschluss setzen wir ROS auf dem Raspberry Pi und mehreren Linux-Rechnern auf und stellen ein WLAN-Netzwerk zwischen den verschiedenen ROS-Komponenten her.
Dieses Netzwerk wird mithilfe von Testausgaben überprüft.

Damit ist eine Grundlage für weitere Anwendungen für ROS auf dem Raspberry geschaffen.\\

{\bf{ optionale Ziele:}}

Bleibt am Ende noch genug Zeit, kompilieren wir ein Steuerungspaket für den ''Volksbots'' auf dem Raspberry. Zur Verbesserung der Usability integrieren wir ein Smartphone ins ROS-Netzwerk. Die Bewegungsrichtung und -geschwindigkeit des Roboters soll durch das Schwenken des Smartphones bestimmt werden.

Haben wir damit Erfolg und wiederum genügend Zeit zur Verfügung, sprechen wir die integrierte Kamera des Roboters an und streamen das Bild mithilfe des Netzwerkes auf einen externen Bildschirm.

Da zukünftige Roboter Ihre Umgebung erkennen und sich sicher im Raum bewegen 
sollen, sehen wir diese Tests als nützlich an.

%Roboter durch einen Raspberry Pi mithilfe von ROS steuern
%Bild einer Webcam durch ROS auf einen Bildschirm streamen

\subsection{Eigene Leistung}

Das ''neue'' an diesem Projekt bezieht sich
%Unsere Leistung bezieht sich
auf die Ablösung eines Laptops durch den kleineren und leichteren Raspberry Pi als ROS-Komponente am ausgewählten Roboter. Desweiteren möchten wir herausfinden, ob die Leistung des Pi für diese Aufgabe ausreicht.

%anstatt Linux PC wird Pi benutzt

\subsection{Aufbau der Arbeit}

\begin{itemize}

\item Recherche über Linux Systeme auf dem Raspberry Pi
\item Passendes Linux System auf Raspberry Pi gesetzt			
\item Recherche über ROS
\item Geeigneten Roboter gewählt (Recherche Verfügbarkeit, Komponenten etc.)
\item Komponenten geschafft (Wlan Stick, Akkupack, etc.)
\item ROS auf Raspberry Pi aufgesetzt
\item Netzwerk zwischen mehreren ROS-Komponenten hergestellt
\item Netzwerk mithilfe von Testausgaben überprüft \\

\end{itemize}

{\bf optional}

\begin{itemize}

\item Robotersteuerung wird zur Zeit implementiert


%Roboter durch einen Raspberry Pi mithilfe von ROS steuern
%Bild einer Webcam durch ROS auf einen Bildschirm streamen

\end{itemize}

\section{Grundlagen}
%6-8 Seiten

\subsection{Raspberry Pi Hardwarekomponenten}
%Zitate von Wikipedia
Der Raspberry Pi ist ein kreditkartengroßer Einplatinencomputer, der von der Raspberry Pi Foundation entwickelt wurde.
Wir benutzen für dieses Projekt das leistungsstärkere Modell B.
%Zitate für die techn. Details?
\parbox{8cm}{
Technische Details:
\begin{itemize}
\item Preis: ca. 35 Euro
\item Prozessor: ARM1176JZF-S (700 MHz)
\item Broadcom VideoCore IV
\item SDRAM: 512 MB
\item Bis zu 16 GPIO-Pins
\item USB-Anschlüsse: 2
\item FBAS, HDMI
\end{itemize}
}
\hfill
\parbox{5cm}{
\includegraphics[width=5cm]{Raspi.jpg}
}
\begin{itemize}
\item 3,5-mm-Klinkenstecker (analog), HDMI (digital)
\item Kartenleser für SD (SDHC und SDXC)/MMC/SDIO
\item 10/100-MBit-Ethernet-Controller 
\item 5 V, 700 mA (3,5 Watt)
\item 5-V-Micro-USB-Anschluss (Micro-B), alternativ 4 x AA-Batterien
\end{itemize}

Für Test und Kontrollausgaben benutzen wir einen am HDMI-Ausgang angeschlossenen Monitor.

Als Stromquelle steht ein Akkupack ......(Firma,Modell) zum Einsatz. 
Hierbei ist wichtig, dass der Raspberry mind. 700 mA, besser 1 A zur Stromversorgung bekommt. 
Bei niedrigerer Amperzahl arbeitet der PC oft nicht zuverlässig.

Ein schneller und großer RAM Speicher ist für unsere Zwecke wichtig, da viele Signale, teils sogar synchron, verarbEin 17-Zoll Monitor wird für die Bildschirmausgabe des Raspberry über HDMI zu DVI benutzteitet werden müssen.

Für die Datenübertragung über das Netz benutzen wir einen Standard 300Mbit/s Wlan-Stick.

Der Raspberry Pi arbeitet mit einer ARM-Prozessorarchitektur.
Diese Architektur wird gerne für embedded Systems, wie PDAs oder Router, eingesetzt.

Sie ist auch auf jedem Smartphone zu finden, da sie den Vorteil einer sehr geringen Leistungsaufnahme bietet. 

%Konfiguration von Linux???
Erwähnenswert ist die Architektur deshalb, weil wir im Laufe des Projektes Schwierigkeiten hatten, auf die wir später eingehen werden
Wir benutzen das auf Linux basierende Betriebssystem Raspbian.
Dabei handelt sich um ein für Raspberry Pi optimiertes open-source Debian-System.
Es enthält viele für die ARM-Architektur vorkompilierte Pakete (über 35.000), dazu auch Features wie etwa eine GUI.

Das System braucht 3GB Speicherbedarf unserer 16GB großen SD-Karte.
Der zusätzliche Speicherplatz wäre nötig, falls man vorhat, Log-Dateien und Ähnliches direkt auf dem Raspberry Pi zu speichern. 
In unserem Fall übernehmen das jedoch die leistungsstärkeren PC's über das ROS Netzwerk.

GPIO (General Purpose Input/Output) ist ein weiteres interessantes Feature des Raspberry Pi.
Es gibt uns die Möglichkeit, jegliche Hardware-Funktionalität anzusteuern.
Beispielsweise können LED-Leuchten oder der Start-Knopf der Kaffeemaschine damit über ROS gesteuert werden.
Wir konnten uns leider jedoch zeitlich bedingt nicht mehr mit dieser Thematik beschäftigen, es bietet aber Anreiz für noch mehr Ansätze und Umsetzungen mithilfe des Raspberry Pi und ROS.


\subsection{Smartphone}

Es wird ein aud Android basierendes Smartphone verwendet.
Wir benutzen auf dem Smartphone eine in einer Projekt-Arbeit entstandene App.
Sie bindet sich an einen ROS-Master (später näher erläutert) und übergibt die X und -Y Koordinaten relativ zu sich weiter.
Diese Werte können für die Steuerung des Motors verwendet werden, indem jeweils ein Wert an eine Radachse geliefert wird. Dieser Wert bildet die Geschwindigkeit des Rades ab.
Zur erleichterten Bedienung stellt uns die App eine GUI zur Verfügung. 

\subsection{Volksbot}

VolksBot ist ein Roboterbaukastensystem.
Mit Hilfe des Baukastens können sehr schnell und preiswert unterschiedlichste Varianten mobiler Roboter hergestellt werden.
Der Volksbot wurde an der Hochschule Weingarten für den RoboCup verwendet.
%Wir brauchen hier mehr Infos, wo wurde er verwendet, Besonderheiten
%Bild
\subsection{ROS}
%Zitate??
ROS ist kein eigentliches Betriebssystem im herkömmlichem Sinne, sondern eine Art strukturierte Kommunikationsschicht.
Die Ziele von ROS können zusammengefasst werden zu:
\begin{itemize}
\item Peer-to-peer (System mit vielen laufenden Prozessen auf verschiedenen Hosts)
\item Tool-based (microkernel Design bestehend aus vielen Komponenten, ähnlich zu Linux)
\item multi-lingual (unterstützt viele Computersprachen, z. B. Python, C++, etc.)
\item thin (Wiederverwendbarkeit von Code steht im Mittelpunkt)
\item open source and free (unter der BSD Lizenz)
\end{itemize}
Es ist das einzig existierende Framework, welches sich auf diese Kriterien spezialisiert.
Den Entwicklern von ROS ging es vor allem um die Vereinfachung, Software für die hohe Zahl an verschiedenen Roboter zu schreiben.
%Zitat
ROS stellt dazu Bibliotheken und Werkzeuge, Hardware Abstraktion, Gerätetreiber, Bibliotheken, Visualisierungen, Nachrichtenvermittlung, Packetverwaltung und andere Komponenten zur Verfügung.

Im Nachfolgenden werden die wichtigsten ROS Core Komponenten kurz beschrieben.

{\bf Topic} Topics können als "named bus" gesehen werden, über welche Nodes Messages verteilen können. Es ähnelt dem Multicasting Prinzip. An Topics kann subscribed (zugehört) oder published (geredet) werden.

{\bf Node} Eine Node ist ein simples Programm, welches jede Funktionalität ermöglicht.  Sie kommunizieren direkt über oben genannte Topics.

{\bf Roscore} Der Roscore kann als Rückgrat des ROS Systems beschrieben werden. Es verwaltet die Registrierung der Nodes und Funktionen in einer Art lookup service für andere Nodes. Es bietet zusätzlich einen Parameter Server, in welchem im Laufenden Betrieb Variablen und Paramater gespeichert werden können und mit welchen gearbeitet werden kann.

{\bf Message} Eine Message kann als Sprache angesehen werden, in welcher Nodes kommunizieren. Es ist eine Struktur aus simplen Parametern.
Beispiel:
\begin{verbatim}
	# This represents a vector in free space. 
	float64 x
	float64 y
	float64 z
\end{verbatim}
Eigene Messages mit den uns bekannten Standardtypen wie int, bool, etc können leicht definiert werden.

{\bf Package} Software in ROS wird mithilfe von Packages verwaltet.
Sie enthalten Nodes, datasets, configuration files, etc.

{\bf Manifest} Sie enthält Meta Informationen über ein ROS Package, wie beispielsweise Abhängigkeiten oder Lizenzinformationen.

{\bf Launchfile} Ein Launchfile ist eine simple XML Datei, welche Informationen über die zu Beginn zu startenden Nodes mit entsprechenden Parametern enthält.
Dies bietet eine komfortablere Alternative, als jedes Node einzeln zu starten.
%Beispiel??
 
{\bf catkin} Catkin ist das seit Groovy in ROS verwendete Workspace-Verwaltungssystem.
Es ist in 4 Spaces unterteilt:
\begin{itemize}
\item Source Space
\item Build Space
\item Devel Space
\item Install Space
\end{itemize}
Catkin bietet uns eine komfortable Möglichkeit über cmake, ROS Packages zu compilieren, Tests an Ihnen durchzuführen und sie schließlich zu installieren.

\section{Problem}

Hauptproblem??

\subsection{Verwandte Arbeiten}

Steffen, Marc
ROS-Tut

\section{Anforderungen}

\begin{itemize}
\item Datenpakete über das ROS-Netzwerk verschicken
\item Beliebige Datentypen verarbeiten
\item Beliebige Linux-Systeme einbinden
\item Überprüfung von Log-Dateien, Kontrollausgaben etc..
\item Robuste und zuverlässige Kommunikation
\item Modulares und flexibles Netzwerk
\item Repositories sollen leicht in Projekte eingebunden werden
\end{itemize}

optional:

\begin{itemize}
\item Roboter soll bei Steuerung in richtige Richtungen fahren
\item Roboter soll sofort auf Richtungsanweisungen reagieren
\item Sensibilität der Steuerung sollte einstellbar sein
\item Steuerung über WLAN
\item Raspberry Pi auf Roboter ohne Kabelzgewirr %haben wir mit WLAN nicht
\item ROS-Core des Raspberry beim Hochfahren automatisch starte
\item Leichte Bedienbarkeit des Roboters

\vspace{0,6cm}

\item Streaming-Bild ruckelfrei verarbeiten
\item unkonvertiert Echtzeitübertragung
\item Ausgabe am externem Bildschirm
\end{itemize}


\section{Lösungsvorschläge}


Es wird ein ROS-Master aufgesetzt. Dieser gilt als Knotenpunkt (vergleichbar mit einem Defaultgateway) und sorgt für die Kommunikationsmöglichkeit im ROS-Netzwerk.

Die Verarbeitung von verschiedenen Datentypen ist durch msg und srv realisierbar
%TODO msg/srv näher erklären

Kontrollausgaben und ähnliches werden durch Horchen an Topics mithilfe von echo und durch Ausgabenbereiche im Quellcode durch das Netzwerk realisiert.
 
%Horchen an Topics mit echo + Ausgabe im Code mithilfe ROS-Stream Fkt.

Die Netzwerkstabilität und -zuverläsigkeit könnte durch Begrenzung der zeitgleichen Datenübertragung umgesetzt werden. Zusätzlich kann man eine Überprüfung der verschickten und erwarteten Datentypen einbinden.

%Verschickte und erwartete Datentypen müssen stimmen
%Richtige Konfiguration des Masters und Clients

Ein Modulares und flexibles Netzwerk, sowie das einfache Einbinden von Repositories ist durch ROS in Kombination mit Catkin gegeben. Die Unterstützung beliebiger Linuxsysteme erfolgt ebenfalls durch ROS. Diese Features müssen allerding zuvor eingebunden werden.
%TODO umschreiben!

%durch ROS und catkin gegeben 
%durch catkin gegeben

optional:

Die Reaktionszeit 
und Steuerung %hinzugefügt, weil oberer abschnitt auskommentiert
wird durch zuverlässige und leistungstarke Hardware, sowie durch einen effizient geschriebenen Code bestimmt. In diesem Fall können wir die Hardware (mit ausßnahme des Pi) aufrüsten. Der Code lässt sich gar nicht oder nur mühsam verändern und austauschen.
%abhängig vom richtigen Code der Motorsteuerung

%abhängig von Raspberry Pi Hardware und effizientem Code und
%guter Hardware

Um bequem mit dem Raspberry arbeiten zu können, ohne jedes mal den ROS-Core starten zu müssen, muss dieser in den Bootvorgang des Raspbian eingebunden werden.

%sollte in Linux beim booten konfiguriert werden

%gegeben durch gute Smartphone App

%\item Streaming-Bild ruckelfrei verarbeiten
%\item unkonvertiert Echtzeitübertragung
%\item Ausgabe am externem Bildschirm

vspace{0,6cm}

Ein sauberes und störungsfreies Übertragungsbild hängt hauptsächlich von der Hardware des Raspberry Pi, aber auch von der Codierung ab. Der Stream soll durch  Umleiten des Bildes über das Netzwerk an einen PC-Monitor realisiert werden. Falls dabei Probleme auftreten sollten, können wir versuchen die Codierung zu bearbeiten oder leistungsverstärkende Hardware an den Pi anschließen.

%abhängig von Raspberry Pi Hardware und Übertragungs-Codierung

%abhängig vom Codex Hardware Komponenten

%nötige Hardware und Netzwerk legen

\section{Bewertung der Lösungen}

Bis jetzt sind unsere Lösungsansätze realativ erfolgreich gewesen. Es könnten jedoch Probleme bei der Steuerung des Roboters über das WLAN-Netzwerk entstehen. Der Grund dafür ist die zusätzliche Datenmenge, die verarbeitet werden muss. 

\section{Implementation}

ROS Netzwerk (topics, nodes, catkin...)


Linux SD Karte Pi


Datenübertragung an Volksbot mithilfe eines USB zu VGA Adapters. Dieser wird per USB an den Raspberry Pi und per VGA an das Steuerungsmodul des Roboters angeschlossen.


Konfiguration Motorsteuerung-Paket und Compilieren


Kontrollausgaben:
\begin{itemize}

\item Horchen am ROS-Topic

\subitem rostopic echo ''Topicbezeichnung''

\item 

\end{itemize}

Netzwerkkomponenten zusammen arbeiten lassen (Hard/-Software)


Wlan -> Teilproblem


%Kamera -> Teilproblem Zeit


\section{Fazit}
1 Seite
\subsection{Zusammenfassung}

hohe Anforderungen für knappen Zeitraum
interessantes, anspruchsvolles Projekt
viel über Linux und Ros gelernt
Hardware- und Netzwerkkonfiguration interessant

\section{Ausblick}

Benutzung des Pi's als Hardware Nodes Lab
Benutzung des Pi's für kleine Roboterprojekte Lab
Immer mehr binary packages für ROS Pi verfügbar -> mehr Möglichkeiten
z.B.: GPIO Interface -> Knöpfe Kaffemaschine, Relays, alle möglichen Signale
auf Hardwareebene

\section{Eigene Leistung}

Durch unser Projekt wird der Laptop am ''Volksbot'' durch einen viel kleineren und leichteren Raspberry Pi ersetzt. Da wir mit ROS arbeiten, kann der Pi mit nur wenig Aufwand auch an anderen Robotern eingesetzt werden.

\section{Quellen}
\begin{verbatim}

http://de.wikipedia.org/wiki/Raspberry_Pi
http://www.amazon.de/Raspberry-Pi-RBCA000-Mainboard-1176JZF-S/dp/B008PT4GGC
http://www.raspbian.org/
http://www.softwareok.de/?seite=faq-System-Allgemein&faq=13
http://de.wikipedia.org/wiki/ARM-Architektur
http://www.rn-wissen.de/index.php/Raspberry_PI:_GPIO
http://www.volksbot.de/index-de.php
http://wiki.ros.org/
http://wiki.ros.org/groovy/Installation/Raspbian
\end{verbatim}

\end{document}